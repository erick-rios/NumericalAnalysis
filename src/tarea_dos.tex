\documentclass[12pt,letterpaper,oneside]{article}
\usepackage[utf8]{inputenc}
\usepackage[spanish]{babel}
\spanishdecimal{.}
\usepackage{amsmath}
\usepackage{amsfonts}
\usepackage{amssymb}
\usepackage{makeidx}
\usepackage{graphicx}
\usepackage{lmodern}
\usepackage{stackrel}
\usepackage{listings}
\usepackage[left=2.5cm,right=2.5cm,top=2.5cm,bottom=2.5cm]{geometry}
\lstset{language=Python, basicstyle=\ttfamily\small, keywordstyle=\color{blue}, commentstyle=\color{gray}, stringstyle=\color{green}}
\title{Temas Selectos de Análisis Numérico\\Práctica 2, parte teórica}
\author{Dra. Úrsula Iturrarán Viveros}
\date{Semestre 2025-1}
\begin{document}
\maketitle
\textbf{Nombres}:
\begin{itemize}
    \item Pablo Rogelio Ramírez Alferes
    \item Erick Jesús Ríos González
\end{itemize}
\paragraph*{Ejercicio 1.} Considera el problema de valor inicial:
\begin{equation}
y'(t) = -y-2t-1,\quad 0\leq t\leq 1,\quad y(0) = 2 \label{eq-1}
\end{equation}

\noindent con solución exacta $y(t) = e^{-t} - 2t + 1$. Encuentra el valor del espaciamiento $h$ que produce al menos dos dígitos decimales de precisión en $t=1$, usando el método de Euler explícito.

\subparagraph*{Solución.}

Primero, recordemos que el método de Euler explícito se define como:
\[
y_{n+1} = y_n + h f(t_n, y_n),
\]
donde \( f(t, y) = -y - 2t - 1 \) en nuestro caso.

Dividimos el intervalo \( [0, 1] \) en \( N \) subintervalos de tamaño \( h = \frac{1}{N} \), con puntos \( t_n = nh \), para \( n = 0, 1, \dots, N \). El valor inicial es \( y(0) = 2 \).

Utilizamos un código de python que se anexa en los archivos adjuntos para encontrar el valor de \( h \) que produce al menos dos dígitos decimales de precisión en \( t = 1 \).
El objetivo es encontrar un valor de \( h \) que asegure al menos dos dígitos decimales de precisión en \( t = 1 \). Para esto, necesitamos comparar la solución numérica \( y_N \) obtenida con el método de Euler en \( t = 1 \), es decir, cuando \( N = \frac{1}{h} \), con la solución exacta \( y(1) = e^{-1} - 2(1) + 1 \).

La solución exacta en \( t = 1 \) es:
\[
y(1) = e^{-1} - 2 + 1 = \frac{1}{e} - 1.
\]
Aproximadamente:
\[
y(1) \approx 0.36788 - 1 = -0.63212.
\]

Para obtener una aproximación numérica usando el método de Euler explícito, el proceso iterativo es el siguiente:
\[
y_{n+1} = y_n + h (-y_n - 2t_n - 1).
\]

Aproximamos el valor de \( y(1) \) para diferentes valores de \( h \), comparando la solución numérica con la exacta.

 Obtuvimos que los siguientes valores de \( h \) producen al menos dos dígitos decimales de precis:
 \begin{table}[h!]
    \centering
    \begin{tabular}{|c|c|c|}
    \hline
    \textbf{h} & \textbf{Aproximación en t=1} & \textbf{Error} \\ \hline
    0.1 & -0.6513215599000002 & 0.019201001071442514 \\ \hline
    0.05 & -0.6415140775914578 & 0.009393518762900177 \\ \hline
    0.01 & -0.633967658726771 & 0.0018470998982133002 \\ \hline
    \end{tabular}
    \caption{Resultados de la aproximación y error en t=1 para diferentes valores de h.}
    \end{table}

 Podemos ver que con \( h = 0.01 \) se obtiene una aproximación con al menos dos dígitos decimales de precisión en \( t = 1 \). Se puede justificar el porqué 0.01 sirve en este caso, recordemos que el método de Euler explícito es de orden 1, es decir que el orden de truncamiento local es menor a $Ch^2$, entonces veamos que tener dos dígitos de diferencia es equivalente a pedir que $\tau_n<\frac{1}{100}$, con nuestra aproximación tenemos que $\tau_n<Ch^2<\frac{1}{100}$, vemos que $\left(\frac{1}{100}\right)^2<\frac{1}{100}$, según nuestras aproximaciones tenemos que $C\approx0.1$, lo cual hace que $Ch^2<h^2<h$, lo cual es lo que queremos.


 \paragraph*{Ejercicio 2.} Encuentra una cota para el error local de truncamiento para el método de Euler explícito aplicado al problema de valor inicial:
 \[
 y'(t) = 2ty,\quad y(1) = 1
 \]
 con solución exacta \(y(t)=e^{t^2-1}\).
 
 \subparagraph*{Solución.} Para el problema de valor inicial dado, se tiene la ecuación diferencial 
 \[
 y'(t) = 2ty,\quad y(1) = 1.
 \]
 Sabemos que el método de Euler explícito tiene la forma:
 \[
 y_{n+1} = y_n + h f(t_n, y_n),
 \]
 donde \(f(t, y) = 2ty\), con paso \(h\), y puntos de malla \(t_n = nh\), para \(n = 0, 1, 2, \dots\).
 
 El error local de truncamiento \( \tau_n \) se define como la diferencia entre la solución exacta \( y(t_{n+1}) \) y un paso del método de Euler, evaluado en el punto \( t_n \):
 \[
 \tau_n = y(t_{n+1}) - y(t_n) - h f(t_n, y(t_n)).
 \]
 Aplicando la expansión de Taylor de \(y(t_{n+1})\) alrededor de \(t_n\):
 \[
 y(t_{n+1}) = y(t_n) + h y'(t_n) + \frac{h^2}{2} y''(t_n) + O(h^3).
 \]
 Sustituyendo la ecuación diferencial \(y'(t) = 2ty(t)\) en la expansión, obtenemos:
 \[
 y(t_{n+1}) = y(t_n) + h (2t_n y(t_n)) + \frac{h^2}{2} y''(t_n) + O(h^3).
 \]
 Para calcular \(y''(t)\), diferenciamos nuevamente la ecuación \(y'(t) = 2ty(t)\):
 \[
 y''(t) = 2y(t) + 2t y'(t) = 2y(t) + 2t (2ty(t)) = 2(1 + 2t^2) y(t).
 \]
 Sustituyendo en la expansión de \(y(t_{n+1})\), obtenemos:
 \[
 y(t_{n+1}) = y(t_n) + h (2t_n y(t_n)) + \frac{h^2}{2} (2(1 + 2t_n^2) y(t_n)) + O(h^3).
 \]
 El método de Euler explícito nos da:
 \[
 y_{n+1} = y_n + h(2t_n y_n).
 \]
 Entonces, el error local de truncamiento es:
 \[
 \tau_n = \frac{h^2}{2} (2(1 + 2t_n^2) y(t_n)) + O(h^3).
 \]
 Para encontrar una cota del error, podemos observar que la función \(2(1 + 2t^2)y(t)\) es continua y acotada en el intervalo de integración considerado. Denotando \( M = \max_{t \in [0, T]} |2(1 + 2t^2) y(t)| \), obtenemos la siguiente cota para el error local de truncamiento:
 \[
 |\tau_n| \leq M \frac{h^2}{2} + O(h^3).
 \]
 Así, una cota para el error local de truncamiento es:
 \[
 |\tau_n| \leq C h^2,
 \]
 donde \(C = \frac{M}{2}\) es una constante que no depende de \(t_n\) ni de \(h\).
 
 Por lo tanto, el error local de truncamiento del método de Euler explícito es del orden de \(O(h^2)\), es decir, el error disminuye cuadráticamente con respecto al tamaño del paso \(h\), como se esperaba.
 

\paragraph*{Ejercicio 3.} Deduce las ecuaciones en diferencias correspondientes al método de Taylor de orden dos para los siguientes problemas de valor inicial:
\begin{itemize}
    \item[a)] $y'=-2ty^2,\quad 0\leq t\leq 1,\quad y(0)=1$
    \item[b)] $y'=3(t-1)^2,\quad 0\leq t\leq 1,\quad y(0)=1$
\end{itemize}

\subparagraph*{Solución.} El método de Taylor de orden dos tiene la siguiente forma general:
\[
y_{n+1} = y_n + h f(t_n, y_n) + \frac{h^2}{2} f_t'(t_n, y_n),
\]
donde \( f_t'(t_n, y_n) \) es la derivada total de \( f(t, y) \) con respecto a \( t \).


\begin{itemize}
    \item Consideremos el problema de valor inicial:
    \[
    y' = -2t y^2,\quad 0\leq t\leq 1,\quad y(0) = 1.
    \]
    Para aplicar el método de Taylor de orden dos, primero identificamos la función \( f(t, y) = -2t y^2 \). Luego, calculamos la derivada de \( f(t, y) \) con respecto a \( t \):
    
    \[
    f_t'(t, y) = \frac{\partial}{\partial t}(-2t y^2) = -2y^2.
    \]
    
    Por lo tanto, la ecuación en diferencias del método de Taylor de orden dos es:
    \[
    y_{n+1} = y_n + h (-2t_n y_n^2) + \frac{h^2}{2} (-2y_n^2),
    \]
    es decir,
    \[
    y_{n+1} = y_n - 2h t_n y_n^2 - h^2 y_n^2.
    \]
    
    
    \item Ahora consideremos el segundo problema de valor inicial:
    \[
    y' = 3(t-1)^2,\quad 0\leq t\leq 1,\quad y(0) = 1.
    \]
    Aquí, \( f(t, y) = 3(t-1)^2 \). La derivada de \( f(t, y) \) con respecto a \( t \) es:
    
    \[
    f_t'(t, y) = \frac{\partial}{\partial t} \left(3(t-1)^2 \right) = 6(t-1).
    \]
    
    Aplicando el método de Taylor de orden dos, la ecuación en diferencias es:
    \[
    y_{n+1} = y_n + h 3(t_n - 1)^2 + \frac{h^2}{2} 6(t_n - 1),
    \]
    o simplificando:
    \[
    y_{n+1} = y_n + 3h (t_n - 1)^2 + 3h^2 (t_n - 1).
    \]
\end{itemize}

Hemos deducido las ecuaciones en diferencias para ambos problemas utilizando el método de Taylor de orden dos. Las fórmulas obtenidas nos permiten aproximar las soluciones numéricas de estos problemas de valor inicial.


El método de Taylor de orden dos es:
\[
y_{n+1} = y_n + h y'(t_n, y_n) + \frac{h^2}{2} y''(t_n, y_n).
\]

\textbf{Parte (a): $y'=-2ty^2$}


Derivamos $y' = -2ty^2$:
\[
y'' = -2y^2 - 4tyy'.
\]
Sustituimos $y' = -2ty^2$ para obtener $y''$ en términos de $y$ y $t$.

\textbf{Parte (b): $y' = 3(t-1)^2$}

Como $y'$ no depende de $y$, simplemente derivamos para obtener $y''$:
\[
y'' = 6(t-1).
\]

\paragraph*{Ejercicio 4.} Obtén las iteraciones de Picard para el problema de valor inicial
\[
y'(t) = 2t(y+1),\quad y(0)=0
\]
y demuestra que convergen a la solución \(y(t)=e^{t^2}-1\).

\subparagraph*{Solución.} El método de iteración de Picard para resolver ecuaciones diferenciales consiste en construir sucesivas aproximaciones de la solución mediante la integral de la ecuación diferencial.

Dado el problema de valor inicial
\[
y'(t) = 2t(y + 1), \quad y(0) = 0,
\]
primero integramos ambos lados:
\[
y(t) = y(0) + \int_0^t 2s(y(s) + 1)\,ds.
\]
Sustituyendo el valor inicial \(y(0) = 0\), obtenemos la siguiente forma iterativa:
\[
y_{n+1}(t) = \int_0^t 2s(y_n(s) + 1)\,ds.
\]

Ahora realizamos las primeras iteraciones de Picard.

Primera iteración (\(y_0(t)\)):
Comenzamos con la aproximación inicial \(y_0(t) = 0\). Sustituyendo en la fórmula iterativa, obtenemos:
\[
y_1(t) = \int_0^t 2s(0 + 1)\,ds = \int_0^t 2s\,ds = t^2.
\]

Segunda iteración (\(y_1(t) = t^2\)):
Sustituyendo \(y_1(t) = t^2\) en la fórmula, obtenemos:
\[
y_2(t) = \int_0^t 2s(t^2 + 1)\,ds = \int_0^t 2s(s^2 + 1)\,ds.
\]
Resolvamos esta integral:
\[
y_2(t) = \int_0^t 2s^3\,ds + \int_0^t 2s\,ds = \left[ \frac{s^4}{2} \right]_0^t + \left[ s^2 \right]_0^t = \frac{t^4}{2} + t^2.
\]

Tercera iteración (\(y_2(t) = \frac{t^4}{2} + t^2\)):
Sustituyendo \(y_2(t)\) en la fórmula, obtenemos:
\[
y_3(t) = \int_0^t 2s\left( \frac{s^4}{2} + s^2 + 1 \right)\,ds.
\]
Resolvamos esta integral:
\[
y_3(t) = \int_0^t \left( s^5 + 2s^3 + 2s \right)\,ds = \left[ \frac{s^6}{6} + \frac{s^4}{2} + s^2 \right]_0^t = \frac{t^6}{6} + \frac{t^4}{2} + t^2.
\]



Observamos un patrón en las iteraciones, veamos que las iteraciones parecen seguir la ecuación $y_n(t)=\sum_{i=1}^n\frac{t^{2i}}{i!}$, demostrémoslo por inducción.
\begin{description}
    \item[Caso base ($n=1$)] Ya demostrado en el cálculo de la primera iteración.
    \item[Hipótesis de inducción $(k\implies k+1)$] Supongamos para una $k\in\mathbb{N}$ tenemos que $y_k(t)=\sum_{i=1}^k \frac{t^{2i}}{i!}$, entonces veamos que:
    \begin{align*}
        y_{k+1}(t)&=\int_0^t2s\left(\sum_{i=1}^n \frac{s^{2i}}{i!}+1\right)\,ds\\
        &=\int_0^t2\sum_{i=1}^n\frac{s^{2i+1}}{i!}\,ds+\int_0^22s\,ds\\
        &=2\left(\sum_{i=1}^n\frac{t^{2(i+1)}}{2i!(i+1)}\right)+t^2\\
        &=\sum_{i=1}^{n+1}\frac{t^{2i}}{i!}
    \end{align*}
\end{description}
Las funciones \(y_n(t)\) se parecen cada vez más a la serie de Taylor de la función \(e^{t^2} - 1\), cuya expansión es:
\[
e^{t^2} - 1 = t^2 + \frac{t^4}{2} + \frac{t^6}{6} + \dots.
\]
Por lo tanto, las iteraciones de Picard convergen a la solución exacta:
\[
y(t) = e^{t^2} - 1.
\]

Las iteraciones de Picard para el problema de valor inicial dado convergen a la solución exacta \(y(t) = e^{t^2} - 1\), como hemos verificado mediante la expansión de las primeras iteraciones y la demostración por inducción.


\paragraph*{Ejercicio 5.} Consideremos la ecuación diferencial \( y'(t) = f(t) \). Demuestra que el método de Runge-Kutta de orden 4 se reduce a la regla de Simpson:
\[
\begin{array}{ccl}
\int_{t_n}^{t_n+h}f(t)dt &=& y_{n+1} - y_n \\
 &\approx & \frac{h}{6}\left(f\left(t_n \right) + 4f\left(t_n+\frac{h}{2}\right) + f\left(t_n+h\right)\right).
\end{array}
\]

\subparagraph*{Solución.}
Primero, recordemos que el método de Runge-Kutta de orden 4 para una ecuación diferencial de la forma \( y'(t) = f(t) \) se define por las siguientes fórmulas:
\[
k_1 = f(t_n),
\]
\[
k_2 = f\left(t_n + \frac{h}{2}, y_n + \frac{h}{2} k_1\right),
\]
\[
k_3 = f\left(t_n + \frac{h}{2}, y_n + \frac{h}{2} k_2\right),
\]
\[
k_4 = f\left(t_n + h, y_n + h k_3\right),
\]
y la aproximación de \(y_{n+1}\) está dada por:
\[
y_{n+1} = y_n + \frac{h}{6}(k_1 + 2k_2 + 2k_3 + k_4).
\]

Dado que la ecuación diferencial que nos ocupa es de la forma \(y'(t) = f(t)\), la solución exacta de \(y(t)\) en el intervalo \([t_n, t_n + h]\) se puede escribir como:
\[
y_{n+1} = y_n + \int_{t_n}^{t_n + h} f(t) \, dt.
\]

Nuestro objetivo es demostrar que el método de Runge-Kutta de orden 4 se reduce a la regla de Simpson cuando se integra \(f(t)\) en el intervalo \([t_n, t_n + h]\). Para ello, analizamos las fórmulas de \(k_1\), \(k_2\), \(k_3\) y \(k_4\).

Dado que \(y'(t) = f(t)\), obtenemos:
\[
k_1 = f(t_n),
\]
\[
k_2 = f\left(t_n + \frac{h}{2}\right), \quad \text{(ya que \( y_n + \frac{h}{2} k_1 \) es irrelevante en esta forma)}
\]
\[
k_3 = f\left(t_n + \frac{h}{2}\right),
\]
\[
k_4 = f(t_n + h).
\]

Sustituyendo estas expresiones en la fórmula de Runge-Kutta, obtenemos:
\[
y_{n+1} = y_n + \frac{h}{6}\left(f(t_n) + 2f\left(t_n + \frac{h}{2}\right) + 2f\left(t_n + \frac{h}{2}\right) + f(t_n + h)\right).
\]
Simplificando:
\[
y_{n+1} = y_n + \frac{h}{6}\left(f(t_n) + 4f\left(t_n + \frac{h}{2}\right) + f(t_n + h)\right).
\]

Esto es precisamente la fórmula de la regla de Simpson para la aproximación de la integral de \(f(t)\) en el intervalo \([t_n, t_n + h]\):
\[
\int_{t_n}^{t_n + h} f(t) \, dt \approx \frac{h}{6} \left(f(t_n) + 4f\left(t_n + \frac{h}{2}\right) + f(t_n + h)\right).
\]

Por lo tanto, hemos demostrado que el método de Runge-Kutta de orden 4 se reduce a la regla de Simpson para la aproximación de la integral de \(f(t)\).

\paragraph*{Ejercicio 6.} Muestra que la solución del problema de valor inicial
\[
y'=e^{-t^2},\quad y(0)=0
\]
es
\[
y(t)=\int_0^te^{-x^2}dx.
\]
Usa el método de Runge-Kutta de orden 4 para aproximar el valor de $y(1)=\int_0^1e^{-x^2}dx$.

\subparagraph*{Solución.}
Primero, observemos que el problema de valor inicial dado es
\[
y' = e^{-t^2}, \quad y(0) = 0.
\]
Sabemos que la derivada de \(y(t)\) es \(e^{-t^2}\), por lo tanto, para encontrar la función \(y(t)\), integramos ambos lados con respecto a \(t\):
\[
y(t) = \int_0^t e^{-x^2} dx + C.
\]
Dado que \(y(0) = 0\), podemos encontrar la constante de integración \(C\). Evaluando la integral en \(t = 0\):
\[
y(0) = \int_0^0 e^{-x^2} dx + C = 0 + C = 0,
\]
lo que implica que \(C = 0\). Así, la solución del problema de valor inicial es:
\[
y(t) = \int_0^t e^{-x^2} dx.
\]

Ahora, utilizaremos el método de Runge-Kutta de orden 4 para aproximar el valor de \(y(1) = \int_0^1 e^{-x^2} dx\).

Sea el problema de valor inicial
\[
y' = e^{-t^2}, \quad y(0) = 0.
\]
Dividimos el intervalo \([0,1]\) en \(N\) subintervalos de tamaño \(h = \frac{1}{N}\) y aplicamos el método de Runge-Kutta de orden 4. Las fórmulas para el método son las siguientes:
\[
k_1 = f(t_n, y_n),
\]
\[
k_2 = f\left(t_n + \frac{h}{2}, y_n + \frac{h}{2} k_1\right),
\]
\[
k_3 = f\left(t_n + \frac{h}{2}, y_n + \frac{h}{2} k_2\right),
\]
\[
k_4 = f\left(t_n + h, y_n + h k_3\right),
\]
\[
y_{n+1} = y_n + \frac{h}{6} (k_1 + 2k_2 + 2k_3 + k_4),
\]
donde \(f(t, y) = e^{-t^2}\).

Aproximamos \(y(1)\) usando \(N = 10\) subintervalos.


\end{document}