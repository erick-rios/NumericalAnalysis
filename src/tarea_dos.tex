\documentclass[12pt,letterpaper,oneside]{article}
\usepackage[utf8]{inputenc}
\usepackage[spanish]{babel}
\spanishdecimal{.}
\usepackage{amsmath}
\usepackage{amsfonts}
\usepackage{amssymb}
\usepackage{makeidx}
\usepackage{graphicx}
\usepackage{lmodern}
\usepackage{stackrel}
\usepackage[left=2.5cm,right=2.5cm,top=2.5cm,bottom=2.5cm]{geometry}
\title{Temas Selectos de Análisis Numérico\\Práctica 2, parte teórica}
\author{Dra. Úrsula Iturrarán Viveros}
\date{Semestre 2025-1}
\begin{document}
\maketitle

\paragraph*{Ejercicio 1.} Considera el problema de valor inicial:
\begin{equation}
y'(t) = -y-2t-1,\quad 0\leq t\leq 1,\quad y(0) = 2 \label{eq-1}
\end{equation}

\noindent con solución exacta $y(t) = e^{-t} - 2t + 1$. Encuentra el valor del espaciamiento $h$ que produce al menos dos dígitos decimales de precisión en $t=1$, usando el método de Euler explícito.

\subparagraph*{Solución.}

El método de Euler explícito se define como:
\[
y_{n+1} = y_n + h f(t_n, y_n),
\]
donde $f(t, y) = -y - 2t - 1$. La condición inicial es $y(0) = 2$.

\textbf{Paso 1: Calcular $f(t, y)$}
\[
f(t, y) = -y - 2t - 1.
\]
Para cada iteración del método de Euler, tenemos:
\[
y_{n+1} = y_n + h(-y_n - 2t_n - 1).
\]

\textbf{Paso 2: Iteraciones}

Iteramos el método de Euler para $t \in [0, 1]$ con diferentes valores de $h$. Dividimos el intervalo en $N$ subintervalos de longitud $h = \frac{1}{N}$.

\textbf{Paso 3: Encontrar el $h$ adecuado}

La solución exacta en $t=1$ es:
\[
y(1) = e^{-1} - 2(1) + 1 \approx -0.6321.
\]
Iteramos el método hasta obtener una aproximación con al menos dos dígitos decimales de precisión.

\paragraph*{Ejercicio 2.} Encuentra una cota para el error local de truncamiento para el método de Euler explícito aplicado al problema de valor inicial:
\begin{equation}
y'(t) = 2ty,\quad y(1) = 1 \label{eq-2}
\end{equation}

\noindent con solución exacta $y(t)=e^{t^2-1}$.

\subparagraph*{Solución.}

La cota del error local de truncamiento para el método de Euler está dada por:
\[
T_n = \frac{h^2}{2} y''(\xi),
\]
donde $\xi \in [t_n, t_{n+1}]$ es un punto intermedio.

\textbf{Paso 1: Derivar $y(t)$}

Sabemos que $y(t) = e^{t^2 - 1}$. Derivamos para obtener $y''(t)$:
\[
y''(t) = 2e^{t^2 - 1} + 4t^2 e^{t^2 - 1}.
\]

\textbf{Paso 2: Sustituir en el error local}

La cota del error local es:
\[
T_n \approx \frac{h^2}{2} \left( 2e^{\xi^2 - 1} + 4\xi^2 e^{\xi^2 - 1} \right).
\]

\paragraph*{Ejercicio 3.} Deduce las ecuaciones en diferencias correspondientes al método de Taylor de orden dos para los siguientes problemas de valor inicial:
\begin{itemize}
    \item[a)] $y'=-2ty^2,\quad 0\leq t\leq 1,\quad y(0)=1$
    \item[b)] $y'=3(t-1)^2,\quad 0\leq t\leq 1,\quad y(0)=1$
\end{itemize}

\subparagraph*{Solución.}

El método de Taylor de orden dos es:
\[
y_{n+1} = y_n + h y'(t_n, y_n) + \frac{h^2}{2} y''(t_n, y_n).
\]

\textbf{Parte (a): $y'=-2ty^2$}

\textbf{Paso 1: Calcular $y''$}

Derivamos $y' = -2ty^2$:
\[
y'' = -2y^2 - 4tyy'.
\]
Sustituimos $y' = -2ty^2$ para obtener $y''$ en términos de $y$ y $t$.

\textbf{Parte (b): $y' = 3(t-1)^2$}

Como $y'$ no depende de $y$, simplemente derivamos para obtener $y''$:
\[
y'' = 6(t-1).
\]

\paragraph*{Ejercicio 4.} Obtén las iteraciones de Picard para el problema de valor inicial
\begin{equation}
y'(t) = 2t(y+1),\quad y(0)=0 \label{eq-3}
\end{equation}

\noindent y demuestra que convergen a la solución $y(t)=e^{t^2}-1$.

\subparagraph*{Solución.}

Las iteraciones de Picard se definen por:
\[
y_{n+1}(t) = y_0 + \int_0^t f(s, y_n(s)) ds,
\]
donde $f(t, y) = 2t(y+1)$.

\textbf{Primera iteración:}

\[
y_1(t) = \int_0^t 2s(0+1) ds = t^2.
\]

\textbf{Segunda iteración:}

\[
y_2(t) = \int_0^t 2s(t^2+1) ds = t^2 + \frac{t^4}{2}.
\]

Iteramos hasta que las soluciones convergen a $y(t) = e^{t^2} - 1$.

\paragraph*{Ejercicio 5.} Consideremos la ecuación diferencial $y'(t) = f(t)$. Demuestra que el método de Runge-Kutta de orden 4 se reduce a la regla de Simpson:
\[
\begin{array}{ccl}
\int_{t_n}^{t_n+h}f(t)dt &=& y_{n+1} - y_n \\
 &\approx & \frac{h}{6}\left(f\left(t_n \right) + 4f\left(t_n+\frac{h}{2}\right) + f\left(t_n+h\right)\right).
\end{array}
\]

\subparagraph*{Solución.}

El método de Runge-Kutta de orden 4 calcula:
\[
k_1 = f(t_n),\quad k_2 = f(t_n + \frac{h}{2}),\quad k_3 = f(t_n + \frac{h}{2}),\quad k_4 = f(t_n + h).
\]
La fórmula de Runge-Kutta es:
\[
y_{n+1} = y_n + \frac{h}{6}(k_1 + 2k_2 + 2k_3 + k_4).
\]
Esto se puede reescribir como una cuadratura de Simpson para la integral de $f(t)$.

\paragraph*{Ejercicio 6.} Muestra que la solución del problema de valor inicial
\[
y'=e^{-t^2},\quad y(0)=0
\]
es
\[
y(t)=\int_0^te^{-x^2}dx.
\]
Usa el método de Runge-Kutta de orden 4 para aproximar el valor de $y(1)=\int_0^1e^{-x^2}dx$.

\subparagraph*{Solución.}

La solución es:
\[
y(t) = \int_0^t e^{-x^2} dx.
\]

Para aproximar $y(1)$, aplicamos el método de Runge-Kutta de orden 4 con $f(t) = e^{-t^2}$. Calculamos las siguientes etapas:
\[
k_1 = e^{-t_n^2},\quad k_2 = e^{-\left(t_n + \frac{h}{2}\right)^2},\quad k_3 = e^{-\left(t_n + \frac{h}{2}\right)^2},\quad k_4 = e^{-\left(t_n + h\right)^2}.
\]
Luego, aplicamos la fórmula:
\[
y_{n+1} = y_n + \frac{h}{6}(k_1 + 2k_2 + 2k_3 + k_4).
\]

\end{document}
